\documentclass[12pt,a4paper,openright,twoside]{report}

\usepackage[italian]{babel}
\usepackage[utf8]{inputenc}
\usepackage{fancyhdr}

% libreria per avere l'indentazione all'inizio dei capitoli
\usepackage{indentfirst}

% libreria per mostrare le etichette
%\usepackage{showkeys}

% libreria per inserire grafici
\usepackage{graphicx}

% libreria per utilizzare font particolari, ad esempio \textsc{}
\usepackage{newlfont}

% librerie matematiche
\usepackage{amssymb}
\usepackage{amsmath}
\usepackage{latexsym}
\usepackage{amsthm}
\usepackage{cite}
\usepackage{listings}
\usepackage{hyperref}
\usepackage[square,numbers,sort]{natbib}
\usepackage{xcolor}

% \bibliographystyle{unsrt}
\bibliographystyle{unsrtnat}

\lstset{
    frame=single,
    breaklines=true
}

\oddsidemargin=30pt
\evensidemargin=20pt

% serve per la sillabazione: tra parentesi vanno inserite come nell'esempio le
% parole che latex non riesce a tagliare nel modo giusto andando a capo.
\hyphenation{sil-la-ba-zio-ne pa-ren-te-si}

% comandi per l'impostazione della pagina
\pagestyle{fancy}
\addtolength{\headwidth}{20pt}
\renewcommand{\chaptermark}[1]{\markboth{\thechapter.\ #1}{}}
\renewcommand{\sectionmark}[1]{\markright{\thesection \ #1}{}}
\rhead[\fancyplain{}{\bfseries\leftmark}]{\fancyplain{}{\bfseries\thepage}}
\cfoot{}

% comando per impostare l'interlinea
\linespread{1.3}

\newcommand{\xstudent}{CHRISTIAN RICCI}
\newcommand{\xsupervisor}{MORENO MARZOLLA}




\begin{document}

% scelta delle dimensioni della pagina
%\setlength{\textwidth}{13.5cm}
%\setlength{\textheight}{19cm}
%\setlength{\footskip}{3cm}

% %%% inizio prefazione %%%

%\textwidth=450pt
\oddsidemargin=25pt

\begin{titlepage}
\begin{center}
    {\Large{\textsc{Alma Mater Studiorum}}}\\
    {\Large{\textsc{Universit\`a di Bologna}}} \\
    {\textsc{Campus di Cesena}}
    % thin and big line
    \rule[0.1cm]{14cm}{0.1mm}
    \rule[0.5cm]{14cm}{0.6mm}
    DIPARTIMENTO DI INFORMATICA – SCIENZA E INGEGNERIA
    Corso di Laurea in Ingegneria e Scienze Informatiche
\end{center}

\vspace{15mm}

\begin{center}
    {\LARGE{\bf STEREOMATCHING SU GPU}} \\
\end{center}

\vspace{15mm}

\begin{center}
     \large{ Elaborato in:\\ High Performance Computing\\}
\end{center}

\vspace{20mm}
\par
\noindent

\begin{minipage}[t]{0.47\textwidth}
    {\large{\bf Relatore:\\ Prof.\\ \xsupervisor}}
\end{minipage}
\hfill
\begin{minipage}[t]{0.47\textwidth}\raggedleft
    {\large{\bf Presentata da:\\ \xstudent}} \end{minipage}
\vspace{20mm}
% TODO anno e sessione
\begin{center}
    \large{\bf Sessione I\\ Anno Accademico 2022-2023}
\end{center}
\end{titlepage}

\clearpage{\pagestyle{empty}\cleardoublepage}

\newpage

% %%% pagina della dedica %%%
\begin{titlepage}
\thispagestyle{empty}
\topmargin=6.5cm
\raggedleft
\large
\em
(Dedica)
\clearpage{\pagestyle{empty}\cleardoublepage}
\end{titlepage}

% %%% introduzione %%%
\chapter*{Introduzione}

\pagenumbering{roman}

% imposta l'intestazione di pagina
\rhead[\fancyplain{}{\bfseries INTRODUZIONE}]{\fancyplain{}{\bfseries\thepage}}
\lhead[\fancyplain{}{\bfseries\thepage}]{\fancyplain{}{\bfseries INTRODUZIONE}}

% aggiunge la voce Introduzione nell'indice
\addcontentsline{toc}{chapter}{Introduzione}

La Stereo Vision è un popolare argomento di ricerca nel campo della Visione Artificiale; esso consiste nell'usare due immagini di una stessa scena, prodotte da due fotocamere diverse, per estrarre informazioni in 3D. L'idea di base della Stereo Vision è la simulazione della visione binoculare umana: le due fotocamere sono disposte in orizzontale per fungere da ``occhi'' che guardano la scena in 3D. Confrontando le due immagini ottenute, l'informa



% %%% indice %%%

\tableofcontents
\rhead[\fancyplain{}{\bfseries\leftmark}]{\fancyplain{}{\bfseries\thepage}}
\lhead[\fancyplain{}{\bfseries\thepage}]{\fancyplain{}{\bfseries INDICE}}
\clearpage{\pagestyle{empty}\cleardoublepage}

%\listoffigures
%\clearpage{\pagestyle{empty}\cleardoublepage}

%\listoftables
%\clearpage{\pagestyle{empty}\cleardoublepage}



% %%% primo capitolo %%%

\clearpage{\pagestyle{empty}\cleardoublepage}
\chapter{Primo Capitolo}

\lhead[\fancyplain{}{\bfseries\thepage}]{\fancyplain{}{\bfseries\rightmark}}
\pagenumbering{arabic}

Primo capitolo

\chapter{Secondo capitolo}

Questo \`e il secondo capitolo.

\chapter{Terzo capitolo}

Questo \`e il terzo capitolo.

\chapter*{Conclusioni}

\rhead[\fancyplain{}{\bfseries CONCLUSIONI}]{\fancyplain{}{\bfseries\thepage}}
\lhead[\fancyplain{}{\bfseries\thepage}]{\fancyplain{}{\bfseries CONCLUSIONI}}

\addcontentsline{toc}{chapter}{Conclusioni}
Queste sono le conclusioni.\\



% %%% bibliografia %%%

\clearpage{\pagestyle{empty}\cleardoublepage}

\rhead[\fancyplain{}{\bfseries BIBLIOGRAFIA}]{\fancyplain{}{\bfseries\thepage}}
\lhead[\fancyplain{}{\bfseries\thepage}]{\fancyplain{}{\bfseries BIBLIOGRAFIA}}

\bibliography{biblio}{}
\bibliographystyle{plain}

\addcontentsline{toc}{chapter}{Bibliografia}

\rhead[\fancyplain{}{\bfseries \leftmark}]{\fancyplain{}{\bfseries\thepage}}
\clearpage{\pagestyle{empty}\cleardoublepage}



\chapter*{Ringraziamenti}
\addcontentsline{toc}{chapter}{Ringraziamenti}
\thispagestyle{empty}
Ringraziamenti.

\nocite{*}

\end{document}
