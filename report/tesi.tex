\documentclass[12pt,a4paper,openright,twoside]{report}

\usepackage[italian]{babel}
\usepackage[utf8]{inputenc}
\usepackage{fancyhdr}

% libreria per avere l'indentazione all'inizio dei capitoli
\usepackage{indentfirst}

% libreria per mostrare le etichette
%\usepackage{showkeys}

% libreria per inserire grafici
\usepackage{graphicx}

% libreria per utilizzare font particolari, ad esempio \textsc{}
\usepackage{newlfont}

% librerie matematiche
\usepackage{amssymb}
\usepackage{amsmath}
\usepackage{latexsym}
\usepackage{amsthm}
\usepackage{cite}
\usepackage{listings}
\usepackage{hyperref}
\usepackage[square,numbers,sort]{natbib}
\usepackage{xcolor}

% \bibliographystyle{unsrt}
\bibliographystyle{unsrtnat}

\lstset{
    frame=single,
    breaklines=true
}

\oddsidemargin=30pt
\evensidemargin=20pt

% serve per la sillabazione: tra parentesi vanno inserite come nell'esempio le
% parole che latex non riesce a tagliare nel modo giusto andando a capo.
\hyphenation{sil-la-ba-zio-ne pa-ren-te-si}

% comandi per l'impostazione della pagina
\pagestyle{fancy}
\addtolength{\headwidth}{20pt}
\renewcommand{\chaptermark}[1]{\markboth{\thechapter.\ #1}{}}
\renewcommand{\sectionmark}[1]{\markright{\thesection \ #1}{}}
\rhead[\fancyplain{}{\bfseries\leftmark}]{\fancyplain{}{\bfseries\thepage}}
\cfoot{}

% comando per impostare l'interlinea
\linespread{1.3}

\newcommand{\xstudent}{Christian Ricci}
\newcommand{\xsupervisor}{Moreno Marzolla}




\begin{document}

% scelta delle dimensioni della pagina
%\setlength{\textwidth}{13.5cm}
%\setlength{\textheight}{19cm}
%\setlength{\footskip}{3cm}

% %%% inizio prefazione %%%

%\textwidth=450pt
\oddsidemargin=25pt

\begin{titlepage}
\begin{center}
    {\Large{\textsc{Alma Mater Studiorum}}}\\
    {\Large{\textsc{Universit\`a di Bologna}}} \\
    {\textsc{Campus di Cesena}}
    % thin and big line
    \rule[0.1cm]{14cm}{0.1mm}
    \rule[0.5cm]{14cm}{0.6mm}
    DIPARTIMENTO DI INFORMATICA – SCIENZA E INGEGNERIA
    Corso di Laurea Magistrale in Ingegneria e Scienze Informatiche
\end{center}

\vspace{15mm}

\begin{center}
    {\LARGE{\bf TITOLO}} \vspace{3mm} \\
    {\LARGE{\bf DELLA}} \vspace{3mm} \\
    {\LARGE{\bf TESI}}
\end{center}

\vspace{15mm}

\begin{center}
     \large{ Elaborato in:\\ ........\\}
\end{center}

\vspace{20mm}
\par
\noindent

\begin{minipage}[t]{0.47\textwidth}
    {\large{\bf Relatore:\\ Prof.\\ \xsupervisor}}
\end{minipage}
\hfill
\begin{minipage}[t]{0.47\textwidth}\raggedleft
    {\large{\bf Presentata da:\\ \xstudent}} \end{minipage}
\vspace{20mm}
% TODO anno e sessione
\begin{center}
    \large{\bf Sessione I\\ Anno Accademico 2015-2016}
\end{center}
\end{titlepage}

\clearpage{\pagestyle{empty}\cleardoublepage}

\newpage

% %%% pagina della dedica %%%
\begin{titlepage}
\thispagestyle{empty}
\topmargin=6.5cm
\raggedleft
\large
\em
(Dedica)
\clearpage{\pagestyle{empty}\cleardoublepage}
\end{titlepage}

% %%% introduzione %%%
\pagenumbering{roman}
\chapter*{Introduzione}

% imposta l'intestazione di pagina
\rhead[\fancyplain{}{\bfseries INTRODUZIONE}]{\fancyplain{}{\bfseries\thepage}}
\lhead[\fancyplain{}{\bfseries\thepage}]{\fancyplain{}{\bfseries INTRODUZIONE}}

% aggiunge la voce Introduzione nell'indice
\addcontentsline{toc}{chapter}{Introduzione}
Questa \`e l'introduzione.



% %%% indice %%%

\tableofcontents

\rhead[\fancyplain{}{\bfseries\leftmark}]{\fancyplain{}{\bfseries\thepage}}
\lhead[\fancyplain{}{\bfseries\thepage}]{\fancyplain{}{\bfseries INDICE}}

\clearpage{\pagestyle{empty}\cleardoublepage}

\listoffigures
\clearpage{\pagestyle{empty}\cleardoublepage}

\listoftables
\clearpage{\pagestyle{empty}\cleardoublepage}



% %%% primo capitolo %%%

\clearpage{\pagestyle{empty}\cleardoublepage}
\chapter{Primo Capitolo}

\lhead[\fancyplain{}{\bfseries\thepage}]{\fancyplain{}{\bfseries\rightmark}}
\pagenumbering{arabic}
Questo \`e il primo capitolo di una tesi scritta in Latex ~\cite{latex}
\section{Prima Sezione}
Questa \`e la prima sezione.

Ora vediamo un elenco numerato:
\begin{enumerate}
\item primo oggetto
\item secondo oggetto
\item terzo oggetto
\item quarto oggetto
\end{enumerate}

\begin{figure}[h]
\begin{center}
\caption[legenda elenco figure]{legenda sotto la figura}\label{fig:prima}
\end{center}
\end{figure}

\section{Seconda Sezione}
Ora vediamo un elenco puntato:
\begin{itemize}
\item primo oggetto
\item secondo oggetto
\end{itemize}

\section{Altra Sezione}
Vediamo un elenco descrittivo:
\begin{description}
  \item[OGGETTO1] prima descrizione;
  \item[OGGETTO2] seconda descrizione;
  \item[OGGETTO3] terza descrizione.
\end{description}

\subsection{Altra SottoSezione}
\subsubsection{SottoSottoSezione}Questa sottosottosezione non viene
numerata, ma \`e solo scritta in grassetto.
\section{Altra Sezione}
Vediamo la creazione di una tabella; la tabella \ref{tab:uno}
(richiamo il nome della tabella utilizzando la label che ho messo sotto):
la facciamo di tre righe e tre colonne, la prima colonna
``incolonnata'' a destra (r) e le altre centrate (c):\\
\begin{table}[h]

\begin{center}
\begin{tabular}{r|c|c}

\hline \hline
$(1,1)$ & $(1,2)$ & $(1,3)$\\
\hline
$(2,1)$ & $(2,2)$ & $(2,3)$\\
\hline
$(3,1)$ & $(3,2)$ & $(3,3)$\\
\hline \hline
\end{tabular}
\caption[legenda elenco tabelle]{legenda tabella}\label{tab:uno}
\end{center}
\end{table}
\section{Altra Sezione}\label{sec:prova}

\subsection{Listati dei programmi}
\subsubsection{Primo Listato}
\begin{verbatim}
        In questo ambiente     posso scrivere      come voglio,
lasciare gli spazi che voglio e non % commentare quando voglio
e ci sarà scritto tutto.
Quando lo uso è meglio che disattivi il Wrap del WinEdt
\end{verbatim}

\clearpage{\pagestyle{empty}\cleardoublepage}
\chapter{Secondo capitolo}

Questo \`e il secondo capitolo.
\section{Prima Sezione}
Questa \`e la prima sezione.

\section{Seconda Sezione}
Questa \`e la seconda sezione.

\clearpage{\pagestyle{empty}\cleardoublepage}
\chapter{Terzo capitolo}

Questo \`e il terzo capitolo.

\section{Prima Sezione}
Questa \`e la prima sezione.

\section{Seconda Sezione}
Questa \`e la seconda sezione.



% inizio parte finale del documento



\bibliography{backMatter/biblio}{}
\bibliographystyle{plain}

% non numera l'ultima pagina sinistra
\clearpage{\pagestyle{empty}\cleardoublepage}
\chapter*{Conclusioni}
% imposta l'intestazione di pagina
\rhead[\fancyplain{}{\bfseries
CONCLUSIONI}]{\fancyplain{}{\bfseries\thepage}}
\lhead[\fancyplain{}{\bfseries\thepage}]{\fancyplain{}{\bfseries
CONCLUSIONI}}

\addcontentsline{toc}{chapter}{Conclusioni} Queste sono le conclusioni.\\

% bibliografia
\clearpage{\pagestyle{empty}\cleardoublepage}

\rhead[\fancyplain{}{\bfseries BIBLIOGRAFIA}]{\fancyplain{}{\bfseries\thepage}}
\lhead[\fancyplain{}{\bfseries\thepage}]{\fancyplain{}{\bfseries BIBLIOGRAFIA}}
% aggiunge l'intestazione di pagina

\addcontentsline{toc}{chapter}{Bibliografia}
% aggiunge la voce Bibliografia nell'indice


\rhead[\fancyplain{}{\bfseries \leftmark}]{\fancyplain{}{\bfseries
\thepage}}
% aggiunge la voce Bibliografia nell'indice
\addcontentsline{toc}{chapter}{Ringraziamenti}

% non numera l'ultima pagina sinistra
\clearpage{\pagestyle{empty}\cleardoublepage}
\chapter*{Ringraziamenti}
\thispagestyle{empty}
Qui possiamo ringraziare il mondo intero!!!!!!!!!!\\
Ovviamente solo se uno vuole, non \`e obbligatorio.
\nocite{*}

%\cleardoublepage
%\addcontentsline{toc}{chapter}{Bibliografia}

%\listoffigures
%\listoftables

\end{document}
